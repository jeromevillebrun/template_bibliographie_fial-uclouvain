\renewbibmacro*{cite:full}{%
  \usebibmacro{cite:full:citepages}%
  \global\toggletrue{cbx:fullcite}%
  \printtext[bibhypertarget]{%
    \usedriver
      {\DeclareNameAlias{sortname}{default}}
      {\thefield{entrytype}}}%
  \usebibmacro{shorthandintro}}



%\renewbibmacro*{cite}{%
%  \usebibmacro{cite:citepages}%
%  \global\togglefalse{cbx:loccit}%
%  \ifciteseen
%    {\ifciteibid
%       {\usebibmacro{cite:ibid}}
%       {\iffieldundef{shorthand}
%          {\usebibmacro{cite:opcit}}%macro opcit au lieu de short.
%          {\usebibmacro{cite:shorthand}}}}
%    {\usebibmacro{cite:full}}}


%la macro qui est utilisée quand la citation a été déjà vue, mais pas juste avant (op. cit.)
%\renewbibmacro*{cite:short}{%
%  \printnames{labelname}%
%  \setunit*{\printdelim{nametitledelim}}%
%  \printtext[bibhyperlink]{%
%    \printfield[citetitle]{labeltitle}}}


%\renewbibmacro*{cite:opcit}{%
%  \printnames{labelname}%
%  \setunit*{\printdelim{nametitledelim}}%
%  \printtext[bibhyperlink]{%
%    \bibstring[\mkibid]{opcit}}}


%\renewbibmacro*{cite:ibid}{%
%  \printtext[bibhyperlink]{\bibstring[\mkibid]{ibidem}}%
%  \ifloccit
%    {\global\toggletrue{cbx:loccit}}%Ceci spécifie qu'on est dans un cas de loccit (référence à une même page). Il ne le fait désormais que si c'est effectivement un loccit (ifloccit).
%    {}}


%\renewbibmacro*{cite:title}{%vient de verbose-trad2
%  \printtext[bibhyperlink]{%
%    \printfield[citetitle]{labeltitle}%
%    \setunit{\printdelim{nametitledelim}}%
%    \bibstring[\mkibid]{opcit}}}

