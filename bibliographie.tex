Fill text.\autocite{unarticle}

\newpage
Ouvrage déjà cité, mais il y a eu d'autres citation entre les deux :
\autocite[15]{poucet}\autocite{test}\autocite[31]{poucet}

\newpage
Plusieurs ouvrages du même auteur : on ajoute le titre complet sans les sous-titres pour lever l'ambiguité
\autocite[146]{genettemim}\autocite{test}\autocite[16--29]{genettefig}\autocite{test}\autocite[103]{genettemim}

\newpage
Deux tomes différents d'un même ouvrage : on mentionne le tome à chaque fois.
\autocite[103]{martin1}\autocite{test}\autocite[71]{martin2}

\newpage
Deux tomes différents d'un même ouvrage avec un titre différent : on mentionne le tome et le titre à chaque fois.
\autocite[115]{remond1}\autocite{test}\autocite{remond2}

\newpage
Deux référence identiques. Les citations se suivent immédiatement → Ibid.
\autocite[41--57]{leclerc}\autocite[41--57]{leclerc}

\newpage
Deux références identiques citées pour des pages différentes. Les citations se suivent immédiatement → Ibid., p.~x.
\autocite[105]{vouilloux}\autocite[80]{vouilloux}

\newpage
Références identiques différent seulement par le tome → ibid., t. x, p. y. %Ici, pour les volumes
On précise le titre et le sous-titre des tomes particuliers s'il y en a.
\autocite[115]{remondx1}\autocite[45]{remondx2}

\newpage
Se suivent immédiatement, ont le même auteur, mais un titre différent → \textsc{Id.}
\autocite[197]{genettexmim}\autocite[91]{genettexfig}

\newpage
Se suivent immédiatement, ont le même auteur, mais un titre différent + Celle pour laquelle on utilise \textsc{Id.} a déjà été citée précédemment → on ajoute op. cit, précédé du titre sans sous-titre.
\autocite[79--93]{genetteyfig}\autocite[146]{genetteymim}\autocite[103]{genetteyfig}

\newpage
Renvoi global, sans citation précise à un article/partie d'ouvrage/partie d'article → Cf.
\autocite[Cf.][]{delatte}




\newpage





\printbibliography[title={Bibliographie}]


% D'après les normes FIAL, la bibliographie doit être organisée comme suit.

% Sources inédites, classées dasn l'ordre alphabétique, par :
% 	- Pays
% 	- Ville
% 	- Dépôt au sein de la ville
% 	- Fonds (ou série pour la France)
% 	- Numéro des pièces concernées
%
% Sources éditées
%
% Travaux, avec possibilité de les classer par thèmes.

% Pour faire les classements, on peut utiliser les types d'entrées, puis des keywords ou des notkeywords. Tout est expliqué dans le paragraphe 3.13.4 de la documentation de biblatex.









% Rappel pour les ibid et op. cit. (vérifier une par une que la règle a été respectée) :
% 
% - l'ouvrage a déjà été cité :
% * Nom Prénom, \emph{op. cit.}, ment_tome_et_titre_du_tome, p. x.
% 
% - un ouvrage du même auteur a déjà été cité:
% * Nom Prénom, titre_sans_le_sous_titre, \emph{op. cit.}, ment_tome_et_titre_du_tome, p. x.
% 
% - un ouvrage du même auteur, de titre différent, a été cité juste avant. À la place de l'auteur :
% * \textsc{Id.}
% 
% - le même ouvrage a été cité juste avant. (numéro de page et mention tome facultatifs) :
% * \emph{Ibid.}, ment_tome_et_titre_du_tome, p. x.



% - Pour un renvoi sans précision de la page :
% * Cf.


%%%%%%%%%%%%%%%%%%%%%%%%%%%%%%%%%%%%%%%%%%%%%%%%%%%%%%%%%%%%%%%%%%%%%%%%%%%%%%%%%%%%%%%%%%%%%%

% monographie/acte de colloque :
%\item \textsc{Nom} Prénom, \textit{Titre}, ment_tome, ment_édition, lieu, éditeur, date, nomb_volumes (collection, n).

% mémoire :
%\item \textsc{Nom} Prénom, \textit{Titre}, Université catholique de Louvain, 2008-2009 [Mémoire_de_maîtrise_en_Histoire]

% bibliographie/dictionnaire/encyclopédie (ne pas indiquer "(éd.)" si le dictionnaire a été fait seul) :
%\item \textsc{Nom} Prénom (éd.), \textit{Titre}, ment_édition, lieu, éditeur, date, num_volume (collection, n).

% référence électronique :
% \item \textsc{Nom} Prénom, «~Titre~», \url{} (consulté le 26 mai 2006).

% source éditée :
%\item \textsc{Nom_aut_intell} Prénom, \textit{Titre}, éd. Prénom_éditeur \textsc{Nom_éditeur}, suite_de_la_réf_selon_la_forme_de_l_édition.

% Recueil de texte (éditeur = personne qui rassemble les textes) :
%\item \textit{Titre_du_recueil}, éd. Prénom_éditeur \textsc{Nom_éditeur}, suite_de_la_réf_selon_la_forme_de_l_édition.

% référence archivistique (pour une page, et puis pour un folio_+_indication_recto_ou_verso) :
%\item \textsc{Ville}, Nom_du_dépôt. \textit{Nom_du_fonds}, n° Cote, p. n.
%\item \textsc{Ville}, Nom_du_dépôt. \textit{Nom_du_fonds}, n° Cote, f° n\up{r/v}.


% article revue :
%\item \textsc{Nom} Prénom, «~Titre_article~», dans \textsc{Nom} Prénom (éd.), \textit{titre_revue}, tome_num, date, p. 5-17.

% article revue si dossier_thém/num_spécial :
%\item \textsc{Nom} Prénom, «~Titre_article~», dans \textsc{Nom} Prénom (éd.), «~Titre_num_special_doss_thém~», dossier spécial/dossier thématique, \textit{titre_revue}, tome_num, date, p. 5-17.

% article journal (périodique) :
%\item \textsc{Nom} Prénom, «~Titre_article~», dans \textit{titre_du_journal}, 25_janvier_2006, p. 5-17.

% article ouvrage collectif :
%\item \textsc{Nom} Prénom, «~Titre_article~», dans \textsc{Nom} Prénom (éd.), \textit{titre_de_l_ouvrage_collectif}, lieu, éditeur, date, num_vol, p. 5-17 (collection, n).

%article bibliographie/dictionnaire/encyclopédie (ne pas indiquer "(éd.)" si le dictionnaire a été fait seul) :
%\item \textsc{Nom} Prénom, «~Titre_article~», dans \textsc{Nom} Prénom (éd.), \textit{titre_dictinonaire}, ment_édition, lieu, éditeur, date, num_volume, p. 5-17 (collection, n).



