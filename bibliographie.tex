\section{Bibliographie}

Texte de remplissage.
\autocite{livre}
Texte de remplissage.
Texte de remplissage.





\printbibliography





%\begin{itemize}
%	\item ceci est une réf.
%\end{itemize}



% Rappel pour les ibid et op. cit. (vérifier une par une que la règle a été respectée) :
% 
% - l'ouvrage a déjà été cité :
% * Nom Prénom, \emph{op. cit.}, ment_tome_et_titre_du_tome, p. x.
% 
% - un ouvrage du même auteur a déjà été cité:
% * Nom Prénom, titre_sans_le_sous_titre, \emph{op. cit.}, ment_tome_et_titre_du_tome, p. x.
% 
% - un ouvrage du même auteur, de titre différent, a été cité juste avant. À la place de l'auteur :
% * \textsc{Id.}
% 
% - le même ouvrage a été cité juste avant. (numéro de page et mention tome facultatifs) :
% * \emph{Ibid.}, ment_tome_et_titre_du_tome, p. x.



% - Pour un renvoi sans précision de la page :
% * Cf.


%%%%%%%%%%%%%%%%%%%%%%%%%%%%%%%%%%%%%%%%%%%%%%%%%%%%%%%%%%%%%%%%%%%%%%%%%%%%%%%%%%%%%%%%%%%%%%

% monographie/acte de colloque :
%\item \textsc{Nom} Prénom, \textit{Titre}, ment_tome, ment_édition, lieu, éditeur, date, nomb_volumes (collection, n).

% mémoire :
%\item \textsc{Nom} Prénom, \textit{Titre}, Université catholique de Louvain, 2008-2009 [Mémoire_de_maîtrise_en_Histoire]

% bibliographie/dictionnaire/encyclopédie (ne pas indiquer "(éd.)" si le dictionnaire a été fait seul) :
%\item \textsc{Nom} Prénom (éd.), \textit{Titre}, ment_édition, lieu, éditeur, date, num_volume (collection, n).

% référence électronique :
% \item \textsc{Nom} Prénom, «~Titre~», \url{} (consulté le 26 mai 2006).

% source éditée :
%\item \textsc{Nom_aut_intell} Prénom, \textit{Titre}, éd. Prénom_éditeur \textsc{Nom_éditeur}, suite_de_la_réf_selon_la_forme_de_l_édition.

% Recueil de texte (éditeur = personne qui rassemble les textes) :
%\item \textit{Titre_du_recueil}, éd. Prénom_éditeur \textsc{Nom_éditeur}, suite_de_la_réf_selon_la_forme_de_l_édition.

% référence archivistique (pour une page, et puis pour un folio_+_indication_recto_ou_verso) :
%\item \textsc{Ville}, Nom_du_dépôt. \textit{Nom_du_fonds}, n° Cote, p. n.
%\item \textsc{Ville}, Nom_du_dépôt. \textit{Nom_du_fonds}, n° Cote, f° n\up{r/v}.


% article revue :
%\item \textsc{Nom} Prénom, «~Titre_article~», dans \textsc{Nom} Prénom (éd.), \textit{titre_revue}, tome_num, date, p. 5-17.

% article revue si dossier_thém/num_spécial :
%\item \textsc{Nom} Prénom, «~Titre_article~», dans \textsc{Nom} Prénom (éd.), «~Titre_num_special_doss_thém~», dossier spécial/dossier thématique, \textit{titre_revue}, tome_num, date, p. 5-17.

% article journal (périodique) :
%\item \textsc{Nom} Prénom, «~Titre_article~», dans \textit{titre_du_journal}, 25_janvier_2006, p. 5-17.

% article ouvrage collectif :
%\item \textsc{Nom} Prénom, «~Titre_article~», dans \textsc{Nom} Prénom (éd.), \textit{titre_de_l_ouvrage_collectif}, lieu, éditeur, date, num_vol, p. 5-17 (collection, n).

%article bibliographie/dictionnaire/encyclopédie (ne pas indiquer "(éd.)" si le dictionnaire a été fait seul) :
%\item \textsc{Nom} Prénom, «~Titre_article~», dans \textsc{Nom} Prénom (éd.), \textit{titre_dictinonaire}, ment_édition, lieu, éditeur, date, num_volume, p. 5-17 (collection, n).



