%Cette page a été concue pour le compilateur xelatex.

%Packages de base
\usepackage{xunicode} %pour l'unicode de XeLaTeX
\usepackage[a4paper]{geometry} %règle les dimensions de la page
\usepackage[french]{babel} %adapte aux spécificités de la langue française
\usepackage[citestyle=verbose-trad3,bibstyle=verbose,sortlocale=auto,sortcase=false,maxnames=3,minnames=1,backref=false,backrefstyle=three,isbn=false,url=false,doi=false,eprint=false,citepages=omit,nosortothers=true,strict=false]{biblatex}
%- citestyle (=numeric par défaut). Mettre sur un style verbose-trad2 ou 3, c'est le style qui ressemble le plus à celui de la FIAL. Il faut mieux mettre verbose-trad3 que verbose-trad2 parce que verbose-trad2 indique pp. X-Y quand il y a plusieurs pages, tandis que trad3 indique p. X-Y.
%- sortlocale (=auto par défaut) fait que les algorithmes alphabétiques de tris unicode s'adaptent à la langue définie par babel (surtout pour ceux qui écrivent en allemand).
%- sortcase (=true par défaut) « Whether or not to sort the bibliography and the list of shorthands case-sensitively. » → pas très clair, il faudra faire des tests pour voir lequel il faut mettre pour que ça trie case-insensitively.
%- maxnames (=3 par défaut). S'il y a plus de 3 auteurs, biblatex tronque la liste (documentation biblatex, p. 48). Il faudra voir comment il fait pour tronquer la liste ! La norme exige qu'on ajoute "e.a." ex : ROULET Eddy e.a.
%- minnames (=1 par défaut). Définit le nombre d'auteurs à afficher quand il y en a trop (documentation biblatex, p. 48).
%- backref (=true par défaut) demande à biblatex d'indiquer à côté de chaque entrée dans la bibliographie les pages auxquelles l'entrée a été citée (documentation biblatex, p. 51).
%- backrefstyle (=three par défaut) indique le formatage des back references (voir documentation biblatex, p. 51).
%- isbn (=true par défaut) pour choisir d'indiquer ou pas l'isbn.
%- url (=true par défaut) pour choisir d'indiquer ou pas l'url. Même s'il est sur false, les entrées de type @online indiquent quand même l'url (car il est obligatoire pour ce type d'entrées.
%- doi (=true par défaut) pour choisir d'indiquer ou pas le doi.
%- eprint (=true par défaut) pour choisir d'indiquer ou pas le eprint.
%- ibidpage (=false par défaut). Mettre sur true pour indiquer que ibidem sans mention de page signifie "même travail + même page" (documentation biblatex p. 61). Ne fonctionne pas pour verbose-trad3.
%- citepages (=permit par défaut). Mettre sur "omit" pour qu'il empêche d'avoir deux mentions de pages dans la citation : le champ pages et la postnote (documentation biblatex, p. 61).
%- nosortothers (=false par défaut). Mettre sur true pour qu'il ne tienne pas compte des "e.a." (plus de 3 noms d'auteurs) dans le classement.
%- strict (=false par défaut) mettre sur false pour éviter des ambiguités dans l'utilisation des ibid.
\usepackage{hyperref} %pour des liens hypertexte. chargé après biblatex.
\usepackage{xurl} %pour avoir des url longs bien mis en page
\usepackage{enumitem} %nécessaire pour personnaliser la forme des listes

%Packages "en plus"
\usepackage{lettrine} %pour faire une première lettre géante
\usepackage{oldgerm} %pour une lettrine stylisée
\usepackage[babel]{csquotes} %commande \enquote{} un texte entre «“”»
\usepackage[final]{pdfpages} %pour ajouter page_de_garde_fial.pdf (\includepdf)

%Réglages
\setlist[itemize]{label=\textbullet} %utilise bullet point pour les listes

%Réglages de bibliographie
\addbibresource{bibliographie.bib} %choisit le fichier contenant les références.
\renewcommand{\newunitpunct}[0]{\addcomma\addspace} %séparateur = ","
\renewcommand{\revsdnamepunct}{\addspace} %pas de "," entre nom et prénom.
\renewcommand{\subtitlepunct}[0]{\adddot\addspace} %point entre titre et sous-titre

%problème : les modifications sont faites pour un style mais pas l'autre.

